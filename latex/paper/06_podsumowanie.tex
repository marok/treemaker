\chapter{Podsumowanie}

Celem naszego projektu inżynierskiego było stworzenie narzędzia do generowania trójwymiarowych modeli drzew. Rezultaty pracy pokrywają w 100\% cele postawione na początku. Wszystkie prace zostały wykonane zgodnie z harmonogramem oraz podziałem zadań.

Sukcesywnie podczas realizacji projektu pojawiało się coraz więcej nowych pomysłów na rozwój aplikacji.

Możliwości rozwoju:
\begin{my_itemize}
	%\item dodatkowe formaty obsługiwane przez eksporter
	\item model fizyczny drzewa
	%\item optymalizacja metod generowania drzewa
	\item dodanie nowych modyfikatorów wpływających na generowany model
	%\item generowanie korzenia
	\item generowanie liści oraz kory
	%\item generowanie kwiatów i owoców
	\item tworzenie kilku modeli kolonizujących tę samą przestrzeń jednocześnie - efekt współzawodniczenia o przestrzeń
	\item poprawienie edytora - dodanie kolejnych możliwości edycji
	\item ustawianie liści do światła
	\item dodanie opcji cofnij/powtórz podczas modyfikowania
	%\item kilkustopniowe generowanie
	%\item w środku drzewa zwykle nie ma liści
	\item inne algorytmy generowania - uwzględniające np. oświetlenie, wpływ wiatru
	%\item animacja wzrostu drzewa
	%\item malowanie "pędzlem" kory, np. dodawanie mchu
	\item optymalizacja modelu pod względem liczby wielokątów
	\item operacje logiczne na koronach
	\item wczytywanie koron z pliku xml
	\item generowanie dowolnej liczby drzew na podstawie pliku xml
	%\item zapisywanie ustawień do xml
	\item inne formaty tekstur niż bmp24bit
	\item filtry kolorów do tekstur
	\item zmienna geometria liści
\end{my_itemize}
