\chapter{Dokumentacja}

\section{Diagramy klas}

\section{Eksport modelu}
Ponieważ format danych używany w programie nie jest zgodny z powszechnie uznanymi formatami modeli trójwymiarowych, aby zachować kompatybilność należy przeprowadzić eksport modelu do 
formatu zewnętrznego. Format .OBJ zgodnie wymaganiami jest obsługiwany przez program Blender.
Został on opracowany przez firmę Wavefront Technologies i ze względu na swoją prostotę stał się szybko
popularny wśród programów do obróbki grafiki trójwymiarowej. Na format składa się tekstowy z rozszerzeniem .obj 
zawierający opis geometrii obiektu, dodatkowo format przewiduje odwołanie do pliku z rozszerzeniem .mtl zawierającym
opis materiałów (kolorów i tekstur).
\subsection{Struktura pliku .obj}
Plik jest podzielony na linie, z czego każda linia moze zawierać:
\begin{itemize}
\item komentarz: zaczynający się od znaku '\#'
\item odwołanie do pliku z materiałami: mtllib [nazwa pliku .mtl]
\item nakaz uzycia materiału: usemtl [nazwa materiału]
\item definicje obiektu: o [nazwa obiektu]
\item definicje grupy: g [nazwa grupy]
\item współrzędne wierzchołka v [x] [y] [z]
\item współrzędne wektora normalnego vn [x] [y] [z]
\item współrzędne tekstury vt [x] [y]
\item definicja trójkąta f [v1/vn1/vt1] [v2/vn2/vt2] [v3/vn3/vt3]
\end{itemize}
Warto nadmienić, iż przy definiowaniu trójkątów podajemy indeksy (numerowane od 1) poszczególnych współrzędnych w znajdujących się w pliku.
Istotne jest również to, iż współrzędne tekstury i wektora normalnego trójkąta są opcjonalne, a sam wektor normalny może być odtworzony poprawnie,
nawet jeśli nie został zawarty w pliku dzięki podaniu współrzędnych wierzchołków zgodnie z ruchem wskazówek zegara.
\subsection{Struktura pliku .mtl}
Plik .mtl może zawierać definicje wielu materiałów. Podobnie jak plik .obj jest to plik tekstowy z informacjami znajdującymi
się w kolejnych liniach mogących zawierać:
\begin{itemize}
\item newmtl [nazwa materiału]: definicja materiału
\item Ka [r] [g] [b]: ambient kolor
\item Kd [r] [g] [b]: diffuse kolor
\item Ks [r] [g] [b]: specular kolorów
\item Ns [x]        : specular cooef
\item d  [x]        : przezroczystosc
\item map\_Ka [nazwa pliku]: tekstura 
\end{itemize}
\subsection{Procedura eksportu}
Model drzewa jest eksportowany w dwóch etapach, jako dwie osobne grupy jednego obiektu. Pierwszą grupę stanowi pień drzewa, drugą natomiast jego liście.
Pozwala to na użycie różnych tekstur dla tych elementów drzewa. Program grupuje współrzędne wierzchołków i tekstur by zmniejszyć rozmiar tworzonego pliku, a następnie
zapisuje je do pliku \{models/tree0.obj\}. Dodatkowo program tworzy plik \{models/tree0.mtl\} zawierający opis materiałów i ścieżki do tekstur. 
\section{Opis interfejsu użytkownika}
\subsection{Okno główne}
{\includegraphics[width=140mm]{images/gui/mainWindow.png}}\\
\vspace{5mm}
Trzeba znalesc cos co ladnie zaznacza i maluje po takim obrazku zeby zaznaczyc widok drzewa dodac literke A i nadole opisac ze A to jest widok drzewa
Opis okna glownego: widok drzewa, panel opcji, toolbar
Opis toolbara, zapisu/odczytu/eksportu

A dalej to juz poszczególne zakładki z opisem, w sumie ze 3 strony
\subsection{Opcje korony}
\subsection{Opcje wyświetlania}
\subsection{Opcje pnia}
\subsection{Opcje liści}
\subsection{Edytor}
\section{Kompilacja projektu}
\subsection{Wymagane biblioteki}
Zgodnie z wymaganiami programi powinien posiadać możliwie małą ilość zależności. System powinien posiadać kompilator GCC oraz bibliotekę GTK+ z rozszerzeniem do obsługi OpenGL.
Poniżej przedstawiono dokładne wersje bibliotek wymagane przez program.
\begin{itemize}
\item GTK+ 2.2
\item GTKGLEXT 1.2
\item MESA 7.8
\item GCC 4.4.4
\end{itemize}
\subsection{Uruchomienie}
Źródła programu zostały załączone na płycie CD, ponadto znajdują się na stronie https://code.google.com/p/treemaker/source/browse/.\\
Po ich pobraniu wystarczy wydać polecenie make w katalogu ze źródłami by rozpocząć kompilację programu. Jeśli przebiegła ona pomyślnie został utworzony
program wykonywalny o nazwie treemaker.
