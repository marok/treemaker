\chapter{Specyfikacja wymagań}
Najlepiej wedlug tego dokumentu:
http://entropy.echelon.pl/miguel/print.php?id=11
Wymagania:\\
\begin{itemize}
\item Kto ma tego uzywac
\item Do czego
\item Po co robimy ten program
\end{itemize}
\begin{enumerate}

\item Aplikacja ma być przeznaczona dla systemu Linux. 
\begin{enumerate}
\item Aplikacja ma być napisana w języku C++.
\item Aplikacja ma działać w trybie graficznym.
\item Kontrolki aplikacji mają wyświetlać podpowiedzi, po najechaniu na nie kursorem.
\item Aplikacja powinna korzystać z jak najmniejszej liczby zewnętrznych bibliotek.
\end{enumerate}
\item Aplikacja ma umożliwiać zmianę parametrów algorytmu generowania.
\item Generowanie modelu nie powinno trwać dłużej niż kilka sekund.
\item Aplikacja ma umożliwiać wybór tekstur liści i kory.
\begin{enumerate}
\item Aplikacja ma obsługiwać format 24bit BMP. 
\item Aplikacja ma umożliwiać wybór kliku tekstur liści.
\item Użytkownik ma mieć możliwość wskazania pliku z teksturą.
\end{enumerate}
\item  Aplikacja ma umożliwiać zapis i odczyt aktualnych ustawień do i z pliku.
\begin{enumerate}
\item Użytkownik ma mieć możliwość wyboru nazwy pliku.
\item Plik z ustawieniami powinien być plikiem tekstowym ASCII.
\end{enumerate}
\item Aplikacja ma umożliwiać eksport modelu drzewa do formatu .OBJ.
\item Użytkownik ma mieć możliwość edycji modelu drzewa w trybie \\WYSWIG.
\begin{enumerate}
\item Wybór aktywnej gałęzi.
\item Zmiana współczynnika grawitacji dla aktywnej gałęzi.
\item Usunięcie aktywnej gałęzi.
\item Wygładzenie aktywnej gałęzi
\item Zmiana rozmiaru liści.
\item Zmiana liczby liści.
\item Zmiana liczby liści danego typu.
\item Zaznaczenie szypułki liścia.
\item Modyfikacja ułożenia tekstury kory.
\end{enumerate}
\end{enumerate}

\section{Ograniczenia aplikacji}
Ponieważ generowanie modelu drzewa jest tematem bardzo szerokim, podczas projektowania
aplikacji przyjęto pewne założenia ograniczające jego funkcjonalność. Było to niezbędne ze
względu na krótki czas realizacji projektu.

Ograniczyliśmi się do:
\begin{itemize}
	\item {małych drzew liściastych}
	\item {wygląd jedynie poglądowy bez zaawansowanego oświetlenia, wygładzania krawędzi}
	\item {teksturowanie pnia tylko jedną teksturą jednocześnie}
	\item {brak korzeni}
	\item {brak owoców i kwiatów}
	\item {brak modelu fizycznego drzewa}
	\item {zamiast tekstury liscia mozna wybrac teksture galezi (?)}
\end{itemize}
