\chapter{Wstęp}

\section{Cel pracy}
\begin{itemize}
	\item{Stworzenie narzędzia pozwalającego na łatwe generowanie trójwymiarowych modeli drzew}
	\item{Zastosowania}
	\begin{itemize}
		\item{Ułatwienie pracy grafikom komputerowym}
		\item{Gry, animacje, wizualizacje i mapy 3D}
	\end{itemize}
\end{itemize}




\section{Funkcjonalność}
\begin{itemize}
	\item{Modyfikacja parametrów algorytmu generującego}
	\begin{itemize}
		\item{kształt drzewa}
		\item{liczba gałęzi}
	\end{itemize}
	\item{Edycja wygenerowanego modelu}
	\begin{itemize}
		\item{usuwanie gałęzi}
		\item{wygładzanie gałęzi}
		\item{zmiana ilośi liści}
		\item{wybor tekstur kory i liści}
	\end{itemize}

	\item{Eksport do formatu obsługiwanego przez program Blender}
\end{itemize}

\section{Zadania do wykonania i podział pracy}
\begin{itemize}
	\item{Algorytm kolonizacji przestrzenii - Łukasz Odzioba}
	\item{Algorytm cząsteczkowy - Łukasz Odzioba}
	\item{Tworzenie modelu drzewa na podstawie struktury zwróconej przez algorytm generowania(obliczanie grubości gałęzi w węzłach, opracowanie i zaimplementowanie sposobu łączenia gałęzi) - Mariusz Okrój}
	\item{Obrys korony drzewa tworzony na podstawie brył - Mariusz Okrój}
	\item{Zapis i odczyt ustawień programu z pliku - Łukasz Odzioba}
	\item{Teksturowanie pnia drzewa - Łukasz Odzioba}
	\item{Teksturowanie liści - Łukasz Odzioba}
	\item{Rozmieszczenie liści na drzewie - Mariusz Okrój}
	\item{Eksport modelu do formatu wspieranego przez program Blender - Łukasz Odzioba}
	\item{Zaimplementowanie edytora pozwalającego obrabiać wygenerowany model - Mariusz Okrój}
\end{itemize}


\section{Harmonogram}

