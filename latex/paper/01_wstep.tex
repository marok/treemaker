\chapter{Wstęp}

\section{Cel pracy}
Celem niniejszego projektu inżynierskiego jest stworzenie narzędzia pozwalającego na łatwe 
tworzenie trójwymiarowych modeli drzew. Narzędzie ma być przeznaczone dla osób zajmujących
się grafiką komputerową.

Ręczne modelowanie skomplikowanej struktury drzew jest bardzo czasochłonne,
dlatego też tworzenie modeli przy użyciu naszej aplikacji ma być przede wszystkim łatwe i szybkie.
W idealnym przypadku do stworzenia modelu potrzebne byłoby kilkanaście sekund. Ponieważ większość
grafików komputerowych posiada swój ulubiony program do pracy, nasza aplikacja powinna zapewnić
łatwy sposób na przeniesienie wygenerowanego modelu do innych narzędzi, nawet tych jeszcze nie istniejących.
Narzędzie może znaleźć zastosowanie podczas tworzenia gier, animacji, wizualizacji architektonicznych
czy map trójwymiarowych.
\\
Trzeba cos tu jeszcze dowalic zeby zapelnic strone


\newpage
\section{Funkcjonalność}
\begin{itemize}
	\item{Modyfikacja parametrów algorytmu generującego}
	\begin{itemize}
		\item{kształt drzewa}
		\item{liczba gałęzi}
	\end{itemize}
	\item{Edycja wygenerowanego modelu}
	\begin{itemize}
		\item{usuwanie gałęzi}
		\item{wygładzanie gałęzi}
		\item{zmiana liczby liści}
		\item{wybór tekstur kory i liści}
	\end{itemize}

	\item{Eksport do formatu obsługiwanego przez program Blender}
\end{itemize}

\section{Zadania do wykonania i podział pracy}
W tabeli poniżej zebrano najważniejsze zadania do wykonania wraz z ich podziałem na osoby odpowiedzialne
za wykonanie poszczególnych etapów.
\begin{center}    
    \begin{longtable}{|p{120mm}|p{16mm}|} \hline
    Algorytm kolonizacji przestrzeni & ŁO \\ \hline
    Zapis i odczyt ustawień programu z pliku  & ŁO \\ \hline
    Teksturowanie modelu drzewa & ŁO \\ \hline
    Eksport modelu do formatu wspieranego przez program Blender & ŁO \\ \hline
    \hline
    Tworzenie geometrii drzewa & MO \\ \hline
    Obrys korony drzewa tworzony na podstawie brył & MO \\ \hline
    Edytor do obróbki wygenerowanego modelu  & MO \\ \hline
    Rozmieszczenie liści na drzewie & MO \\ \hline
    \hline
    Opracowanie literatury & ŁO MO  \\  \hline
    Interfejs użytkownika & ŁO MO \\ \hline
    Renderowanie modelu drzewa & ŁO MO \\ \hline
    Stworzenie przykładowych modeli drzew & ŁO MO \\ \hline
    \end{longtable}
\end{center}


\newpage
\section{Harmonogram}
Wstępny harmonogram prac został ustalony na początku tworzenia projektu. W trakcie pracy był często
modyfikowany, a końcowy uproszony harmonogram przedstawia tabela umieszczona poniżej. Prace nad mniejszymi fragmentami
tworzonego systemu trwały w czasie całego okresu tworzenia, i nie zostały zawarte w tabeli. 
\\\indent Wszystkie etapy podzielono na trzy fazy, w ramach których zadania do wykonania zostały posortowane
po czasie ich rozpoczęcia, a następnie zakończenia.\\ \\
    \indent Faza analizy i projektowania
	\begin{longtable}{|p{85mm}|p{42mm}|} \hline
	

    Zapoznanie się z istniejącymi narzędziami do generowania drzew &
    1.10.2011 -- 7.10.2011
    
    \\ \hline
    Wybranie algorytmów generowania drzewa do implementacji&
    1.10.2011 -- 7.10.2011
    \\ \hline

    Opracowanie literatury&
    1.10.2011 -- 31.10.2011
    \\ \hline

    Specyfikacja wymagań funkcjonalnych&
    15.10.2011 -- 25.10.2011
    \\ \hline
    
    Projektowanie interfejsu użytkownika &
    15.10.2011 -- 1.11.2011
    \\ \hline

    Wybór formatu eksportu modelu &
    7.11.2011 -- 14.11.2011
        \\ \hline

    
    \end{longtable}
	
     Faza implementacji
    \begin{longtable}{|p{85mm}|p{42mm}|} \hline
    Implementacja algorytmu kolonizacyjnego &
    8.10.2011 -- 15.10.2011
    \\ \hline

    Implementacja algorytmów tworzenia geometrii drzewa &
    16.10.2011 -- 31.11.2011 
    \\ \hline

    Implementacja renederowania modelu drzewa &
    16.10.2011 -- 31.11.2011
    \\ \hline

    Implmentacja interfejsu użytkownika &
    1.11.2011 -- 30.11.2011
    \\ \hline

    Implementacja edytora modelu &
    15.11.2011 -- 30.11.2011
    \\ \hline

    Implementacja eksportu modelu do formatu .OBJ&
    15.11.2011 -- 30.11.2011
    \\ \hline
	
   
    
    \end{longtable}
	
    Faza wykończeniowa
    \begin{longtable}{|p{85mm}|p{42mm}|} \hline
    Testowanie aplikacji
    & 1.12.2011 -- 7.12.2011
    \\ \hline

    Przygotowanie przykładowych modeli drzew
    & 1.12.2011 -- 7.12.2011
    \\ \hline
	
    Opracowanie dokumentacji projektu
    & 1.11.2011 -- 7.12.2011
    \\ \hline
	
    \end{longtable}
