\chapter{Wstęp teoretyczny}



\section{Historia i stan obecny dziedziny}
Dynamiczny rozwój grafiki komputerowej obserwujemy od roku 1950 kiedy to w Massachusets Institute of Technology [3] powstał pierwszy
komputer wyposażony w grafoskop. Obecnie większość użytkowników nie potrafi sobie wyobrazić komputera bez graficznego interfejsu.
Coraz większe możliwości nowoczesnych komputerów są motorem napędowym rozmaitych badań nad komputerowymi modelami świata rzeczywistego.
Prekursorem w dziedzinie wizualizacji drzew był polski matematyk Stanisław Ulam, przedstawiajac w 1962r[5] drzewo jako samoorganizującą się
strukturę. Jego model wykorzystujący automaty komórkowe opierał się na rywalizacji węzłów  drzewa o przestreń.
Honda zaproponował w 1971r. modelowanie drzew jako stuktury rekursywnej opisanej
zbiorem parametrów takich jak kąt rozgałęzień, czy stosunek długości między kolejnymi poziomami rekurencji[4]. Dopiero w latach osiemdziesiatych
możliwości uwczesnych komputerów pozwoliły na trójwymiarową wizualizację drzew, czego przykładem może być praca Bloomenthala z 85r [6]
przedstawiająca proces modelowania klonu. Podejście do tematu możemy podzielić na dwie grupy:
\begin{itemize}
\item od ogółu do szczegółu, czyli modelowanie z zadanymi parametrami wyglądu drzewa
\item od szczegółu do ogółu, czyli opisanie pewnym zbiorem parametrów budowy drzewa
\end{itemize}
Problem stworzenia modelu drzewa możemy podzielić na stworzenie modelów jego składowych np gałęzi, kory, liści, czy kwiatów i późniejsze
ich połączenie w jedną całość. Na chwilę obecną istnieją zaawansowane metody generowania zarówno całych drzew, jak i poszczególnych jego części.

\section{Algorytm kolonizacyjny}
Algorytm kolonizacji przestrzeni dalej nazywany algorytmem kolonizacyjnym 
opiera się na biologicznym aspekcie rywalizacji roślin o dostępną 
wokół nich przestrzeń. Po raz pierwszy został zaprezentowany w 2007 roku [1], będąc
rozszerzeniem na przestrzeń trójwymiarową metody generowania liści przy wykorzystaniu
systemów cząsteczkowych [2]. Podstawową ich ideą było umieszczanie cząstek w obrysie liścia,
a następnie śledzenie ich ruchu w kierunku szypułki uwzględniając wzajemne przyciąganie cząstek.
Z biologicznego punktu widzenia miało to uzasadnienie jako śledzenie trasy transportu substancji niezbędnych do życia rośliny.
W podejściu kolonizacyjnym symuluje się iteracyjny wzrost gałęzi drzewa, aż do wykorzystania całego dostępnego miejsca.
Słowniczek:
węzeł drzewa - odcinek w przestrzeni trójwymiarowej reprezentujący część gałęzi
atraktor - pkt w przestrzeni trójwymiarowej, do którego osiągniecia dążą węzły drzewa
korona - podzbiór punktów przestrzeni trójwymiarowej
Dane algorytmu:
korona, zbior parametrow:
dist(a,b) - funkcja zwracająca odległość euklidesową między dwoma punktami przestrzeni 
\begin{verbatim}
  rozmieść atraktory wewnątrz korony drzewa
  dodaj do drzewa wezel glowny
  dopóki istnieją atraktory:
    dla kazdego atraktora a :
      wyznacz najbliższy węzeł drzewa n
       jesli dist(n,a) <=di
        oblicz wektor n->a
      dodaj nowy wezel do drzewa (n-a)
  dla kazdego atraktora
    jesli istnieje wezel drzewa w odleglosci <=dk
      usun atraktor
\end{verbatim}
Parametry:
di - influence distance
dk - kill distance
D - node length
points - liczba atraktorów

Ponieważ istnieją parametry uniemożliwiające zakonczenie algorytmu np dk<<D warunek konca obliczen na potrzeby implementacji został
zastąpiony sprawdzeniem, czy w poprzedniej iteracji został utworzony jakiś nowy węzeł drzewa.
\section{Tworzenie geometrii modelu}

Przyblizanie galezi za pomoca uogolninych rur
Obliczanie grubosci galezi
Przykład kodu źródłowego:

\begin{verbatim}
    
    @Override
    public String toString() {
        return "x: " + x + " y: " + y + " z: " + z;
    }

\end{verbatim}

\section{Teksturowanie}